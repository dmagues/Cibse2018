\section*{Agradecimientos}
%This work was supported by DIUDA 22316 research project of Universidad de Atacama, Chile and partially supported by the \textquotedblleft Laboratorio Industrial en Ingenier�a del Software Emp�rica (LI2SE)\textquotedblright, and the \textquotedblleft An�lisis Experimental de la Efectividad de Test Driven Development en Funci�n de las Caracter�sticas Particulares de los Programadores a Nivel Personal\textquotedblright~research projects of Universidad de las Fuerzas Armadas ESPE, Ecuador.

%Este trabajo ha sido subvencionado por el proyecto de investigaci�n denominado: "Laboratorio Industrial en Ingenier�a del Software Emp�rica (LI2SE)" de la ESPE y adem�s cont� con la financiaci�n del "Proyecto DIUDA 22316" de la Universidad de Atacama.

%La presente investigaci�n ha sido financiada por los proyectos del Ministerio de Educaci�n, Cultura y Deporte de Espa�a FLEXOR (TIN2014-52129-R) y TIN2014-60490-P, el proyecto eMadrid-CM (S2013 / ICE-2715) y el \textquotedblleft Proyecto DIUDA 22316 \textquotedblright de la Universidad de Atacama. Adem�s, cont� con la financiaci�n del proyecto de investigaci�n \textquotedblleft Laboratorio Industrial en Ingenier�a del Software Emp�rica (LI2SE) \textquotedblright de la ESPE.

La presente investigaci�n ha sido financiada por los proyectos del Ministerio de Educaci�n, Cultura y Deporte de Espa�a FLEXOR (TIN2014-52129-R) y TIN2014-60490-P, el proyecto eMadrid-CM (S2013 / ICE-2715), el \textquotedblleft Proyecto DIUDA 22316 \textquotedblright de la Universidad de Atacama, y el proyecto de investigaci�n \textquotedblleft Laboratorio Industrial en Ingenier�a del Software Emp�rica (LI2SE) \textquotedblright de la ESPE.