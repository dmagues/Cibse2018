
%%%%%%%%%%%%%%%%%%%%%%% file typeinst.tex %%%%%%%%%%%%%%%%%%%%%%%%%
%
% This is the LaTeX source for the instructions to authors using
% the LaTeX document class 'llncs.cls' for contributions to
% the Lecture Notes in Computer Sciences series.
% http://www.springer.com/lncs       Springer Heidelberg 2006/05/04
%
% It may be used as a template for your own input - copy it
% to a new file with a new name and use it as the basis
% for your article.
%
% NB: the document class 'llncs' has its own and detailed documentation, see
% ftp://ftp.springer.de/data/pubftp/pub/tex/latex/llncs/latex2e/llncsdoc.pdf
%
%%%%%%%%%%%%%%%%%%%%%%%%%%%%%%%%%%%%%%%%%%%%%%%%%%%%%%%%%%%%%%%%%%%


\documentclass[runningheads,a4paper]{llncs}
%\documentclass[a4paper]{llncs}

\usepackage{amssymb}
\setcounter{tocdepth}{3}
\usepackage{graphicx}

\usepackage[latin1]{inputenc}
\usepackage{float}
\usepackage{colortbl}
\usepackage{hyperref}
\usepackage{comment}
\usepackage{enumerate}
\usepackage{booktabs}
\usepackage{diagbox}
\usepackage{multirow}
\usepackage{pdflscape}
\usepackage{fancyhdr}
\usepackage{tabulary}
\usepackage{soul}
\usepackage{rotating}
\usepackage{framed}
\usepackage[normalem]{ulem}
\usepackage{subfigure}

% Geovanny Need For R Swave
\usepackage{graphicx}
\usepackage{standalone}
\usepackage{import}
\graphicspath{{tex-files/}} 

% correct bad hyphenation here
\hyphenation{op-tical net-works semi-conduc-tor pre-sen-cia}
\hyphenation{la-bo-ra-to-rio cum-pli-mien-to pri-ma-rio u-sa-do rea-li-za-do}
\hyphenation{pri-ma-rios u-sa-bi-li-dad re-troali-men-ta-c}

% Para escribir notas
\newcommand{\odnote}[1]{\textcolor{red}{[OD:#1]}}
\newcommand{\rodnote}[1]{\textcolor{blue}{[RF:#1]}}

\usepackage{url}
\urldef{\mailsa}\path|daniel.magues@estudiante.uam.es|
\urldef{\mailsb}\path|erfonseca@espe.edu.ec|
\urldef{\mailsc}\path|john.castro@uda.cl|
\urldef{\mailsd}\path|silvia.acunna@uam.es|

%\urldef{\mailsc}\path|erika.siebert-cole, peter.strasser, lncs}@springer.com|    
\newcommand{\keywords}[1]{\par\addvspace\baselineskip
\noindent\keywordname\enspace\ignorespaces#1}

\renewcommand{\tablename}{Tabla}
\renewcommand{\refname}{Referencias}

\begin{document}

\mainmatter  % start of an individual contribution

% first the title is needed
\title{Evaluation Usability Techniques Adopted in Agile Development Processes}

% a short form should be given in case it is too long for the running head
\titlerunning{Evaluation Usability Techniques}

% the name(s) of the author(s) follow(s) next
%
% NB: Chinese authors should write their first names(s) in front of
% their surnames. This ensures that the names appear correctly in
% the running heads and the author index.
%
\author{Daniel~A.~Mag\"{u}es$^{1}$
%\thanks{Please note that the LNCS Editorial assumes that all authors have used the western naming convention, with given names preceding surnames. This determines the structure of the names in the running heads and the author index.}%
\and Efra\'in R. Fonseca C.$^{2}$ \and John W. Castro$^{3}$ \and Silvia T. Acu\~na$^{1}$}
%
\authorrunning{Daniel~A.~Mag\"{u}es et al.}
% (feature abused for this document to repeat the title also on left hand pages)

% the affiliations are given next; don't give your e-mail address
% unless you accept that it will be published
\institute{$^{1}$Escuela Polit\'{e}cnica Superior, Universidad Aut\'{o}noma de Madrid, Madrid, Espa\~na
\\$^{2}$Departamento de Ciencias de la Computaci�n, Universidad de las Fuerzas Armadas ESPE, Sangolqu\'i, Ecuador
\\$^{3}$Departamento de Ingenier�a Inform�tica y Ciencias de la Computaci�n, Universidad de Atacama, Copiap\'o, Chile
\\
%\url{http://www.springer.com/lncs}}
\mailsa\\	
\mailsb\\
\mailsc\\
\mailsd\\
}
%
% NB: a more complex sample for affiliations and the mapping to the
% corresponding authors can be found in the file "llncs.dem"
% (search for the string "\mainmatter" where a contribution starts).
% "llncs.dem" accompanies the document class "llncs.cls".
%

%\toctitle{Lecture Notes in Computer Science}
%\tocauthor{Authors' Instructions}
\maketitle

\begin{abstract}
%John: se ha realizado peque�as mejoras en el abstracto pricipalmente en las conclusiones
\textit{Context: }Integration of agile software development process (ASDP) with user-centred design (UCD) has awakened the interest of researchers during the last decade. However, to the best of our knowledge there is no clarity about what techniques or methods are being applied and how to apply them. \textit{Aim: } Our purpose is to identify Usability Evaluation Methods (UEM) which are being incorporated into ASDP and practicalities of its use. \textit{Method: }Based on primary studies of a previous systematic mapping study about application of usability methods in ASDP, we conducted an analysis of such methods through a compiled catalog of usability methods made by researchers of Software Engineering. Finally, we determine the form or type of application of each method into ASDP. \textit{Results: }The agile community show great interest in the incorporation of usability evaluation methods. However, there is no consensus on how to apply these techniques within the agile community. We identify four UEMs within the literature that are the most used in ASDP: Usability Test, Heuristic Evaluation, Thinking Aloud and Focus Group. \textit{Conclusions:} The analysis of usability evaluation methods and its use provided us specific methods that are being incorporated and an overview of activities performed to incorporate such methods into ASDP, which will be an input to define a unified framework of usability evaluating methods in ASDP.
\keywords{Agile Software Development, Usability, Usability Techniques, Usability Evaluation, User-centred Design}
\end{abstract}

\section{Introducci�n}\label{sec-introduccion}
La integraci�n de los procesos de desarrollo de software �giles (PDSA) (por ejemplo XP \cite{Beck1999} y SCRUM \cite{Cohn2009}) con el proceso de Desarrollo Centrado en el Usuario (DCU) ha sido objeto de investigaci�n en los �ltimos a�os \cite{Humayoun2011,Salah2014b,Wale-Kolade2015}. Hemos percibido que este creciente inter�s se debe a que el DCU permite entender las necesidades reales de los usuarios del sistema, y c�mo sus objetivos y actividades pueden ser soportados por el software para mejorar la usabilidad y la satisfacci�n de los usuarios que interact�an con el sistema. Sin embargo, dichas caracter�sticas no son tomadas en cuenta en los PDSA durante el dise�o de la interacci�n del usuario con el sistema software \cite{Adikari2013}. Los desarrolladores �giles experimentan frustraci�n realizando las tareas relacionadas con la evaluaci�n de la usabilidad debido a la brecha de integraci�n existente entre los PDSA y el DCU \cite{Wale-Kolade2014}. La usabilidad es un atributo de calidad en el uso de los sistemas software y no solo radica en la apariencia de la interfaz de usuario, sino en c�mo el usuario interact�a con el sistema \cite{Juristo2007}.\par

%John: Se modifica segun comentario de revisor en la primera oracion
Por un lado, en los PDSA la evaluaci�n de las funcionalidades se hace en cada historia de usuario y en cada iteraci�n. Seg�n Cohn \cite{Cohn2009}, esta estrategia permite asegurar la calidad en cada historia de usuario ya que el equipo incluye tareas de evaluaci�n como test funcionales y control de calidad realizadas por los miembros del equipo y test de aceptaci�n de usuario o \textit{sprint reviews} en SCRUM, cuyo objetivo es validar el cumplimiento de los requisitos desarrollados en las historias de usuario. Por otro lado, el DCU nos permite poner al usuario en el centro de las actividades de an�lisis de requisitos, dise�o y evaluaci�n para mejorar la usabilidad del producto final \cite{Humayoun2011}. Los profesionales de DCU buscan involucrar al usuario aplicando m�todos de DCU en un proceso colaborativo e iterativo que permita validar y evolucionar el producto con los usuarios en un contexto real \cite{Chamberlain2006,Fox2008,Anwar2014}. Por lo tanto, los profesionales de DCU y los equipos que utilizan los PDSA podr�an ser compatibles, permitiendo as� mejores experiencias de usuario.\par

Sin embargo, la evaluaci�n en los PDSA no prioriza la usabilidad desde el punto de vista de la disciplina de DCU. Esto ocurre cuando, por ejemplo, los requisitos de usabilidad se postergan por las necesidades de los clientes con mayor valor de negocio y no se consideran los distintos tipos de usuarios finales del sistema, dificultando la identificaci�n de los problemas de usabilidad que puedan tener los usuarios finales con menor experiencia \cite{Kane2003,Sohaib2010}. El DCU, en cambio, tiene como uno de sus principios entender a todos los perfiles de los usuarios \cite{Sohaib2010}. Muchos autores de la disciplina de la Interacci�n Persona-Ordenador (IPO) afirman que los profesionales de DCU deben acoplar su mentalidad al proceso �gil \cite{Barksdale2009,Brown2013,Seyam2015}. Sin embargo, dicho acoplamiento no resulta del todo sencillo por dos razones. En primer lugar, muchos profesionales de DCU trabajan en equipos separados del equipo de desarrollo �gil, lo que hace que la cultura sea diferente entre ellos. En segundo lugar, muchos profesionales han tenido que desarrollar sus propias estrategias para mantener las pr�cticas de DCU en sinton�a con la organizaci�n, mientras �sta adopta los PDSA \cite{Bertholdo2014}. As�, los profesionales de DCU deben adaptar m�todos de usabilidad siguiendo su experiencia, y muchas de estos m�todos de usabilidad requieren de tiempos y recursos que un proceso �gil no permite. Adem�s, los procesos �giles no proveen gu�as para este tipo de adaptaciones. \par

%John: Se modifica este parrafo en la ultima sentencia
El presente trabajo de investigaci�n realiza un primer acercamiento en la identificaci�n de como otros proyectos �giles han incorporado M�todos de Evaluaci�n de la Usabilidad (UEM por sus siglas en ingl�s). Este art�culo analiza los resultados de una revisi�n de la literatura previa \cite{Magues2016a} que busca identificar los m�todos de usabilidad que est�n siendo incorporadas en los PDSA y c�mo se est�n incorporando. El an�lisis detallado de c�mo se est�n adaptando las UEM adoptadas en los PDSA ser� realizado como trabajo futuro y posteriormente proponer un marco de trabajo formalizado y sistematizado para que los practicantes de desarrollo �gil tengan una gu�a para el uso de estos m�todos. Las contribuciones de nuestro trabajo son las siguientes:

%[EL PROBLEMA QUE SE MENCIONAN EN EL P�RRAFO ANTERIOR, PARA MI GUSTO, NO REPRESENTA UN PROBLEMA, SINO AL CONTRARIO ES YA UNA SOLUCI�N. HABLAR DE UN PRIMER ACERCAMIENTO SIGNIFICA UNA ACCI�N A TOMAR. CREO QUE EL PROBLEMA SIGUE SIENDO LA FALTA DE USABILIDAD ASOCIADO AL DESARROLLO �GIL. HAGO ESTE COMENTARIO, AUNQUE POSIBLEMENTE NO SIGNIFIQUE QUE CAMBIEMOS ALGO, PORQUE ES IMPORTANTE QUE DIFERENCIEMOS CLARAMENTE EL PROBLEMA QUE ESTAMOS ABORDANDO, LA PROPUESTA DE SOLUCI�N Y LAS ACCIONES ASOCIADAS A LA MISMA]

\begin{itemize}
\item Identificamos m�todos de usabilidad relacionadas con actividades de Evaluaci�n que algunos autores han incorporado en los PDSA y analizamos si estas m�todos han sido utilizadas tal como lo prescribe la IPO o si han requerido adaptaciones al contexto del desarrollo �gil para poder ser aplicadas.
\item Determinamos los impedimentos que motivaron las adaptaciones para los UEMs relacionados con las actividades de Evaluaci�n.
\end{itemize}

Lo que resta del art�culo est� organizado de la siguiente manera. En la Secci�n 2 presentamos brevemente el cat�logo de m�todos de la IPO que hemos usado como base para determinar que UEMs relacionadas con la Evaluaci�n est�n siendo usadas por la comunidad �gil. La Secci�n 3 describe el m�todo de investigaci�n. La Secci�n 4 reporta el uso de UEMs relacionados con las actividades de evaluaci�n en los PDSA. La Secci�n 5 presenta la discusi�n de c�mo han sido incorporadas estos m�todos por los PDSA. Finalmente, la Secci�n 6 reporta nuestras conclusiones y trabajos futuros.\par


\section{Trabajos Relacionados}\label{sec-relacionados}
Para investigar de manera general, entre otros aspectos, las t�cnicas de usabilidad que pueden ser incorporadas en los PDSA, algunos investigadores han realizado estudios emp�ricos \cite{Broschinsky2008,Grigoreanu2013,Hodgetts2005}, mientras que otros pocos han estudiado la literatura \cite{Salah2014b,Brhel2015}. El objetivo de nuestra investigaci�n es conocer el estado actual de la usabilidad en los PDSA desde una perspectiva m�s amplia. Para analizar los beneficios de la usabilidad en los PDSA e identificar qu� t�cnicas de usabilidad relacionadas con la actividad de Evaluaci�n est�n siendo adoptadas y c�mo est�n siendo adoptadas, hemos llevado a cabo un \textit{Systematic Mapping Study} (SMS) sobre la usabilidad y los PDSA. Este SMS se diferencia de otros trabajos similares \cite{Salah2014b}\cite{Brhel2015}\cite{DaSilva2011} porque busca identificar en la literatura existente trabajos con propuestas que permitan incorporar un conjunto de t�cnicas de usabilidad para integrar los PDSA con el DCU de una manera formalizada y sistematizada. Aspecto que ninguno de los tres trabajos similares mencionados anteriormente considera.\par

La literatura sobre la integraci�n de t�cnicas de usabilidad en los PDSA est� compuesta por un conjunto de publicaciones desacopladas que estudian diferentes aspectos del tema. En este conjunto, se identifican dos problemas. En primer lugar, existen pocos trabajos que estudian como un todo el tema y reportan cu�l es el estado actual de la integraci�n de los PDSA con el DCU \cite{Brhel2015,DaSilva2011,Sohaib2010}. En segundo lugar, no existen propuestas formalizadas de incorporaci�n de las t�cnicas de usabilidad en los PDSA \cite{Ferreira2012,Hodgetts2005,Hussain2009b} que establezcan, para cada t�cnica, gu�as sobre como incorporarlas.

Con el fin de investigar qu� t�cnicas de usabilidad se est�n utilizando en los PDSA para producir sistemas usables, es preciso identificar primero el universo de las t�cnicas de la IPO. Esto no es una labor sencilla, porque existe una amplia gama de t�cnicas de usabilidad de la IPO, donde la misma t�cnica puede ser referida de manera diferente dependiendo del autor y puede existir m�s de una variante para la misma t�cnica. Afortunadamente, otros investigadores de la ingenier�a de software (IS) ya han realizado el trabajo de compilar un cat�logo de t�cnicas de la IPO, como es el caso de Ferr� \cite{Ferre2005}, quien compil� una lista de t�cnicas de las fuentes m�s reconocidas en el �rea de la IPO. A continuaci�n, se presenta un breve resumen de este cat�logo que deber�a ayudar a los lectores a seguir el resto del documento y a comprender qu� t�cnicas de usabilidad se est�n aplicando actualmente en los PDSA y c�mo se est�n adoptando.\par

Seg�n Ferr� \cite{Ferre2005}, las actividades m�s representativas del proceso de la IPO son la especificaci�n de contexto de uso, especificaciones de usabilidad, desarrollo de concepto del producto, prototipado, dise�o de la interacci�n y evaluaci�n de la usabilidad. Ferr� \cite{Ferre2005} traza estas actividades (y sus respectivas t�cnicas asociadas) a las principales etapas de desarrollo de la IS. En algunos casos, las actividades de la IPO se integran en las actividades existentes de la IS. Por ejemplo, la actividad de especificaciones de usabilidad se puede integrar en la especificaci�n de requisitos. En otros casos, sin embargo, otras actividades que normalmente no se llevan a cabo en un proceso de desarrollo no centrado en el usuario, como el dise�o de la interacci�n, tienen que ser agregadas. Estas actividades adicionales ser�n referidas como se acostumbra en la comunidad de la IPO. Las actividades de la IPO han sido mapeadas teniendo en cuenta las etapas de desarrollo de la IS: Ingenier��a de requisitos, dise�o y evaluaci�n. La actividad de la IPO dise�o de la interacci�n se ha asignado a la actividad de dise�o de la IS. Es importante mencionar, que las t�cnicas de la IPO en el cat�logo est�n clasificadas en tres grandes grupos: t�cnicas relacionadas con las actividades de la ingenier�a de requisitos, t�cnicas relacionadas con las actividades de dise�o y t�cnicas relacionadas con las actividades de evaluaci�n. El an�lisis de las t�cnicas de usabilidad relacionadas con las actividades de la Ingenier�a de Requisitos han sido reportadas en \cite{Magues2016c}. En este trabajo nos centramos en las actividades de Evaluaci�n, es decir, en el estudio de c�mo est�n siendo incorporadas por los PDSA las t�cnicas de usabilidad relacionadas con esta actividad.\par

%%[ME EST� FALTANDO UN PAR DE L�NEAS EN LAS QUE SE DIGAN QUE VAMOS HACER CON ESTO DE FERR�. ESTO PODR�A SER A�ADIDO A LA �LTIMA ORACI�N].
%% Justo antes del p�rrafo de Ferr� se comenta porque hacemos menci�n a este catalogo.

\section{M�todo de Investigaci�n}\label{sec-SMS}

A partir de los estudios primarios obtenidos en un SMS previo, hemos identificado las UEMs que se est�n utilizando en los PDSA. Posteriormente, hemos comparado lo reportado en los estudios primarios con lo prescrito por la IPO, para cada t�cnica identificada con el objetivo de identificar como se est�n incorporando en los PDSA. En la secci�n \ref{sec-synthesis} describimos el proceso seguido para realizar tal identificaci�n. En esta secci�n presentamos un resumen del SMS realizado y los detalles del mismo pueden ser consultados en \cite{Magues2016a}. Las bases de datos electr�nicas (BBDD) utilizadas en el SMS fueron \textit{Scopus}, \textit{ACM Digital Library} e \textit{IEEE Xplore}. La revisi�n abarca los trabajos publicados hasta el 15 de octubre del 2015. Inicialmente, al menos dos de los investigadores que participa en el SMS realiza una revisi�n inicial individual de literatura y propone un grupo de estudios que responden al objetivo planteado en el SMS, sin descuidar la fiabilidad del origen de dichos estudios. Esta revisi�n tiene como prop�sito identificar t�rminos generales o comunes entre los estudios (y sus sin�nimos), para conformar la cadena de b�squeda. La Tabla \ref{table:keywords} ilustra la cadena de b�squeda utilizada en el SMS.\par 

%[CREO QUE DEBEMOS SER M�S PRECISOS EN DESCRIBIR EL M�TODO DE INVESTIGACI�N, DADO QUE TAL CUAL EST� EL P�RRAFO ANTERIOR SE ENTIENDE QUE SE HIZO UNA SMS, LO CUAL NO ES AS�, SOLO SE UTILIZ� DICHA SMS. DEBEMOS INDICAR EXPL�CITAMENTE QUE EL M�TODO CONSISTE DE TALES Y TALES ACTIVIDADES: LA PRIMERA, QUE NO SE C�MO DESCRIBIRLA, FUE HACER ALGO CON LA SMS. NO SE SI ANALIZAMOS LA SMS EXTRAJIMOS CIERTAS PARTE O ALGO AS�. LA SEGUNDA ACTIVIDAD FUE UTILIZAR LO EXTRA�DO DE LA SMS DE LA MANO CON EL TRABAJO DE FERR� PARA HACER ALGO. LUEGO RESUMIOS EL SMS, SEGUIDO DE UNA DESCRIPCI�N DE LA SEGUNDA ACTIVIDAD. IMAGINO QUE ALGO AS� DEBER�A SER ESTA SECCI�N, QUE ADEM�S RECUERDO QUE ME DIJISTE QUE ES LO QUE SE HIZO.]

Los criterios de inclusi�n de la revisi�n de la literatura indican que los trabajos de investigaci�n deben mencionar alg�n tema relacionado con la usabilidad y el desarrollo �gil. La b�squeda se dividi� en dos etapas. Durante la primera etapa, examinamos el t��tulo, las palabras clave y el resumen para inspeccionar los 409 documentos recuperados de las tres BBDD, de los cuales seleccionamos 172 como documentos posiblemente relevantes. S�lo se seleccionaron los trabajos escritos en ingl�s y cuyo resumen o t�tulo mencionaba alg�n tema relacionado con la integraci�n de procesos �giles y usabilidad, alg�n tema relacionado con la ingenier�a de usabilidad o m�todos de la IPO, o relacionado con el proceso de DCU. Los trabajos fueron rechazados si no hac�an menci�n alguna a la integraci�n de procesos �giles y usabilidad o al proceso de DCU. Durante la segunda etapa, le�mos el resumen, la introducci�n y las conclusiones para determinar si el documento describ�a cualquier tipo de integraci�n entre procesos �giles y usabilidad. Finalmente, como resultado de la segunda etapa, se recuperaron un total de 31 art�culos relevantes (estudios primarios).\par
  
\begin{table}
\centering
\caption{Palabras clave utilizadas para la cadena de b�squeda}
\begin{tabular}{rcl}
\hline
\multicolumn{3}{c}{\textbf{Palabras Clave}} \\
\hline
``usability'' \textbf{OR} & \multirow{9}{*}{\textbf{AND}} & ``agile development'' \textbf{OR} \\
``usability method'' \textbf{OR} & & ``agile software development'' \textbf{OR} \\
``usability technique'' \textbf{OR} & & ``agile method'' \textbf{OR} \\
``usability engineering'' \textbf{OR} & & ``agile process'' \textbf{OR} \\
``usability practice'' \textbf{OR} & & ``agile project'' \textbf{OR} \\
``user centered design'' \textbf{OR} & & ``agile practice'' \textbf{OR} \\
``user-centered design'' \textbf{OR} & & ``extreme programming'' \textbf{OR} \\
``user interaction'' \textbf{OR} & & ``scrum'' \\
``user experience'' & & \\
\hline
\end{tabular}

\label{table:keywords}
\end{table}
  
Todos los miembros del equipo de investigaci�n ayudaron a definir la estrategia de b�squeda. Dos investigadores dise�aron y realizaron las b�squedas y extrajeron los datos. Posteriormente, los resultados fueron discutidos en reuniones a las que asistieron todos los miembros del equipo. Luego de analizar los estudios primarios, los clasificamos por tipo de integraci�n en 4 grupos: Integraci�n de procesos, integraci�n de pr�cticas, integraci�n de equipos e integraci�n con apoyo de la tecnolog�a. En este trabajo de investigaci�n, realizamos el an�lisis de los trabajos del grupo Integraci�n de Pr�cticas, que identifica la adopci�n de m�todos de usabilidad en los PDSA y viceversa.\par

Los estudios primarios que estudiamos para analizar los m�todos de usabilidad adoptadas en los PDSA se refieren a la integraci�n de pr�cticas (31 publicaciones). La raz�n por la que seleccionamos este tipo de integraci�n radica en que, como no se requieren cambios profundos en los procesos, ser�a m�s factible integrar los m�todos de usabilidad en los PDSA, ya que no se requerir�a de un gran esfuerzo para capacitar a los equipos de los PDSA, ni ser�a necesario realizar una inversi�n adicional para incluir nuevos roles en el equipo. Despu�s de discutir cada art�culo en las reuniones a las que asistieron todos los miembros del equipo de investigaci�n, determinamos qu� UEMs se han adoptado en los PDSA, proporcionando la descripci�n dada por el autor a cada m�todo. Estos m�todos se analizaron comparando la forma en que se aplic� el m�todo en los PDSA con la recomendaci�n original establecida por el �rea de la IPO. Se necesitaron varias iteraciones para identificar estos m�todos. El proceso se complic� por el hecho de que los autores de los estudios primarios no utilizaron los nombres habituales del m�todo utilizados com�nmente en la literatura de la IPO. Por lo tanto, fue necesario leer el documento m�s a fondo con el fin de identificar el m�todo a la que los autores se refieren. A trav�s de la descripci�n de los m�todos adoptados en los PDSA, el uso de un cat�logo de m�todos de la IPO \cite{Ferre2005} y la consulta con expertos, se identificaron los m�todos de usabilidad reportadas en la literatura que han sido adoptadas en los PDSA. Dos expertos en m�todos de usabilidad participaron en este proceso de identificaci�n. El resultado de este proceso dio lugar a una clasificaci�n preliminar de los diferentes m�todos de usabilidad relacionadas con las actividades de evaluaci�n adoptadas en los PDSA.\par

A continuaci�n, analizamos c�mo se adoptaron los m�todos de usabilidad en las actividades de evaluaci�n de los PDSA. Como resultado de este an�lisis, encontramos que algunas t�cnicas fueron adoptadas de acuerdo a las recomendaciones de la IPO, mientras que otras necesitaron adaptaciones para poder ser incorporadas. Para cada una de las t�cnicas adoptadas por los PDSA, identificamos las adaptaciones realizadas (discutidas en la Secci�n 5). Esta identificaci�n result� ser uno de los procesos m�s complejos, debido al hecho de que los autores de los estudios primarios solo describen la t�cnica de usabilidad adoptada, es decir estos autores no compararon la t�cnica de usabilidad con lo prescrito por la IPO. Esta comparaci�n fue necesaria para poder identificar las adaptaciones realizadas a las t�cnicas.\par

%\begin{table}[htbp]
%	\caption{T�cnicas de Usabilidad de la IPO Relacionadas con la Evaluaci�n Adoptadas por los PDSA}
%	\centering
%		\includegraphics[width=1.1\textwidth]{./images/TablaResumenTecnicas.png}
%	%\label{fig:TabResumCriticas}
%	\label{table:eval}
%\end{table}

%\begin{figure}[htbp]
%	\centering
%		\includegraphics[width=0.7\textwidth]{./images/Propuestas.eps}
%	\captionsetup{justification=centerlast} %Divide el t�tulo en dos l�neas centradas
%	\caption{Propuestas de Tipo de Soluciones Encontradas en los Estudios Primarios}
%	\label{fig:Propuestas}
%\end{figure}

%\begin{table*}%[!t]
\begin{table}
  \caption{T�cnicas de Usabilidad de la IPO Relacionadas con la Evaluaci�n Adoptadas por los PDSA.}
  \label{table:eval}
  \centering
  \resizebox{\textwidth}{!}{%
\begin{tabular}{|cc|c|c|>{\centering}p{3cm}|c|c|}
\hline
%1 header
\rowcolor[gray]{0.8} \multicolumn{2}{|c|}{\textbf{Tipo de}} & \textbf{T�cnica de} & \textbf{Nombre seg�n} & \textbf{Nombre seg�n} & \textbf{Ref.} & \textbf{Tipo de} \\
\rowcolor[gray]{0.8} \multicolumn{2}{|c|}{\textbf{Actividad}} & \textbf{la IPO}  & \textbf{Autores IPO} & \textbf{Autores PDSA} &  & \textbf{Aplicaci�n}  \\
\hline
%126
\multicolumn{2}{|c|}{Evaluaci�n por} & Evaluaci�n & Evaluaci�n & Heuristic Evaluation & \cite{Anwar2014}\cite{Faulring2012}\cite{Losada2012}\cite{Wale-Kolade2014} & Pura \\
 \cline{5-7}
%127
\multicolumn{2}{|c|}{Expertos} & Heur�stica & Heur�stica & Heuristic Evaluation & \cite{Kane2003}\cite{Sohaib2011} & Adaptada \\
 \cline{3-7}
%136
 & & Inspecciones & Inspecciones & Inspection Evaluation & \cite{Larusdottir2014} & Adaptada \\
 \cline{5-7} 
%137
 & & & & Usability Inspections & \cite{DaSilva2015} & Adaptada \\
 \cline{3-7} 
%138
 & & Recorridos & Recorridos & Cognitive Walkthrough & \cite{Faulring2012}\cite{Losada2012} & Pura \\
 \cline{5-7} 
%139
 & & Cognitivos & Cognitivos & Informal Cognitive & \cite{Grigoreanu2013} & Adaptada \\
 & & & & Walkthrough &  &  \\
 \cline{3-7}  
%142
 & & Evaluaci�n por & Evaluaci�n por & Expert Evaluations & \cite{Duechting2007} & Adaptada \\
 \cline{5-7} 
%143
 & & Expertos & Expertos & Usability Expert & \cite{Hussain2009c} & Pura \\
 & & & & Evaluations & & \\ 
 \hline  
%145
\multicolumn{2}{|c|}{Test de} & Pensar en & Pensar en & Simplified Thinking Aloud & \cite{Kane2003} & Adaptada \\ 
 \cline{5-7}
%146
% & & & & Thinking Aloud & \cite{Losada2012}\cite{Losada2013b} & Pura \\ 
\multicolumn{2}{|c|}{usabilidad} & Voz Alta & Voz Alta & Thinking Aloud & \cite{Losada2012}\cite{Losada2013b} & Pura \\ 
 \cline{5-7}
%148
 & & & & Thinking Aloud & \cite{Sohaib2011} & Adaptada \\ 
 \cline{3-7}  
%155
 & & Medici�n del & M�tricas de & Measuring User Performance & \cite{Larusdottir2014} & Adaptada \\ 
 \cline{5-7} 
%156
 & & Rendimiento & Rendimiento & Performance Measurement & \cite{Losada2012} & Pura \\ 
 \cline{3-7}
%160
 & & Test Usabilidad & Laboratorios de & Laboratory Usability & \cite{Hussain2009c} & Pura \\ 
 & & en Laboratorio & Usabilidad & Testing &  & \\  
 \cline{3-7} 
%172
 & & Eval. por Control & Eval. por Control & Remote Synchronous & \cite{Lizano2014} & Adaptada \\ 
 & & Remoto & Remoto & User Testing &  & \\  
 \cline{3-7}
%174
 & & Test de & Test de & Usability Testing & \cite{Ambler2006}\cite{Wale-Kolade2014}\cite{Sy2007}\cite{Wolkerstorfer2008} & Adaptada \\
 & & Usabilidad & Usabilidad &  &  & \\ %%copiada 
 \cline{5-7}  
%175
 & &   &   & Usability Testing & \cite{Anwar2014}\cite{Ferreira2007}\cite{Hudson2003}\cite{Larusdottir2010}\cite{Oevad2015}\cite{Hussain2009b} & Pura \\ 
 \cline{5-7} 
%176
 & &   &   & Test UIs with Users in Paper & \cite{Beyer2004} & Adaptada \\ 
 & &   &   & with Mock-up Interviews & & \\  
 \cline{5-7}
%177
 & &   &   & Informal Usability Testing & \cite{Dayton2009} & Adaptada \\ 
 \cline{5-7}  
%180
 & &   &   & Automated Usability & \cite{Hussain2009b} & Adaptada \\ 
 & &   &   & Evaluations & & \\  
 \cline{5-7} 
%181
 & &   &   & Rapid Iterative Testing & \cite{Hussain2009c} & Adaptada \\ 
 \cline{5-7} 
%182
 & &   &   & Low and High Fidelity  & \cite{Hussain2009d} & Adaptada \\ 
 & &   &   & Prototypes Based & & \\  
 & &   &   & Usability Tests & & \\   
 \cline{5-7}  
%183
 & &   &   & Usability Sessions & \cite{Illmensee2009} & Adaptada \\ 
 \cline{5-7}
%184
 & &   &   & Low-cost Rapid & \cite{Kushniruk2015} & Adaptada \\ 
 & &   &   & Usability Testing  & & \\  
 \cline{5-7} 

%189
 & &   &   & Rapid Iterative Test & \cite{McGinn2013} & Adaptada \\ 
 & &   &   & and Evaluation Method  & & \\  
 \cline{5-7} 
%186
 & &   &   & Micro Tests & \cite{Nielsen2012} & Adaptada \\ 
 \cline{5-7} 
%190
 & &   &   & Extended Unit tests & \cite{Wolkerstorfer2008} & Adaptada \\ 
 & &   &   & for Automated & & \\
 & &   &   & Usability Evaluation  &  & \\  
 \hline
%194
\multicolumn{2}{|c|}{Estudio de} & Cuestionario y & Cuestionarios & Questionnaires & \cite{Duechting2007} & Pura \\  
 \cline{4-7}
%197
\multicolumn{2}{|c|}{Seguimiento} & Encuestas & Encuestas & Surveys & \cite{Larusdottir2014} & Pura \\
  de Sistemas & &  &  &  &  & \\ %%copiada
 \cline{3-7} 
%201
 & & Focus Groups & Focus Groups & Focus Groups & \cite{Anwar2014}\cite{Losada2012}\cite{Losada2013b} & Pura \\  
 \cline{5-7}  
%202
 & & & & Workshops & \cite{Jia2012}\cite{Oevad2015}\cite{Cajander2013} & Pura \\  
 \cline{3-7} 
%215
 & & Retroalimentaci�n & Retroalimentaci�n & Feedback from User & \cite{Larusdottir2014} & Adaptada \\  
 & & del Usuario & del Usuario & Surrogates &  & \\  
 \cline{5-7}
%216
 & & & & Asking User Their Opinions & \cite{Larusdottir2014} & Adaptada \\  
\hline
\end{tabular}}
\end{table}

%[CREO QUE ESTA ES LA PARTE M�S FLOJA DEL ART�CULO, YA QUE NO EST� CLARA LA PARTE METODOL�GICA; POR LO TANTO, ES EN ESTA �RTE EN LA QUE HAY QUE PONER M�S �NFASIS]

\section{Uso de M�todos de la IPO Relacionadas con la Evaluaci�n en los Desarrollos �giles}\label{sec-synthesis}

Hemos encontrado que la comunidad �gil ha adoptado una serie de m�todos de usabilidad en sus proyectos de desarrollo. Hemos clasificado los m�todos de usabilidad adoptadas en dos grupos. El primer grupo incluye todos los m�todos que se han adoptado tal cual, es decir, se han aplicado seg�n lo recomendado por la IPO. El segundo grupo incluye los m�todos que han tenido que ser adaptadas para poder su adopci�n. Con el fin de identificar los m�todos adoptadas en los PDSA, hemos examinado los estudios primarios pertenecientes al grupo de Integraci�n de Pr�cticas. Luego, le�mos cada documento cuidadosamente para identificar los nombres de todas los m�todos reportados por el autor y su respectiva descripci�n. Es importante mencionar que algunos autores reportan la adopci�n de m�s de un m�todo de usabilidad. Luego de leer sus respectivas descripciones, clasificamos cada m�todo adoptada en los PDSA de acuerdo con el cat�logo de m�todos de la IPO \cite{Ferre2005}. Esta tarea se llev� a cabo conjuntamente con expertos en el �rea como parte de las sesiones de trabajo destinadas a identificar qu� m�todos del cat�logo hab�an sido adoptadas por los PDSA. Para identificar estos m�todos, no consideramos los nombres dados por los autores de los estudios primarios, por dos razones. En primer lugar, porque estos autores no usan los nombres de los m�todos com�nmente aceptados por el �rea de la IPO para cada m�todo. En segundo lugar, porque los autores de los estudios primarios al no ser expertos en usabilidad, ni en el catalogo de m�todos, pod�an asignar de manera err�nea los nombres a las mismas. Por tanto, solo consideramos la descripci�n dada por el autor que adopt� el m�todo. Obs�rvese que algunas de los m�todos y su respectiva descripci�n requer�an de un an�lisis m�s profundo para identificar el m�todo de la IPO en el cat�logo. Encontramos que los PDSA han adoptado 13 de las 39 (33.33\%) UEMs. Los 39 m�todos se encuentran agrupadas en tres grandes grupos: Evaluaci\'on por expertos, Test de Usabilidad y Estudio de seguimiento de sistemas instalados. En el grupo de Evaluaci\'on por expertos, la t\'ecnica m�s adoptada por los autores es la Evaluaci\'o Heur\'istica. En los m�todos del grupo Test de Usabilidad, la que lleva el mismo nombre, Test de Usabilidad, ha sido la m�s adoptada. En el grupo de Estudio de seguimiento encontramos que el m�todo de Cuestionarios es la m�s adoptada. La Tabla \ref{table:eval} resume los m�todos de la IPO relacionadas con las actividades de Evaluaci�n que identificamos est�n siendo adoptadas por los PDSA. Para cada t\'ecnica, se especifica el tipo de actividad de Evaluaci�n con la que se relaciona, el nombre gen\'erico de la t\'ecnica, el nombre dado por los diferentes autores en la literatura de la IPO, el nombre utilizado por los autores de los PDSA, las referencias correspondientes y el nivel de adopci�n (pura que significa tal cual como lo prescribe la IPO o con adaptaciones).\par

%[ESTO QUE ESTAMOS CONTANDO AQU� ES M�S M�TODO QUE OTRA COSA, PORQUE DECIMOS QUE ACTIVIDADES SE LLEVARON A CABO. POSIBLEMENTE SEA UNA BUENA IDEA INTEGRAR ESA SECCI�N EN LA SECCI�N M�TODO DE INVESTIGACI�N Y HACERLA UNA SOLA]

\section{Resultados y Discusi�n}\label{sec-results}

En esta secci�n, discutiremos c\'omo los PDSA han adoptado las t\'ecnicas de usabilidad relacionadas con las actividades de Evaluaci\'on. Es importante mencionar que todas las t\'ecnicas de usabilidad identificadas y que han sido adoptadas en proyectos de desarrollo �gil, corresponden a las t\'ecnicas reportadas en la literatura y obtenidas a partir de nuestro SMS. Hemos clasificado todas y cada una de las t\'ecnicas de usabilidad que han sido adoptadas por los desarrollos de los PDSA. Esta clasificaci�n la hemos realizado seg�n un cat�logo de t�cnicas de la IPO \cite{Ferre2005}, comparando c�mo se aplic� la t�cnica con lo prescrito por la IPO.\par 

Como resultado de esta clasificaci�n hemos encontrado que las t�cnicas se est�n adoptando de dos maneras. En primer lugar, las t�cnicas de usabilidad que han sido aplicadas puras, es decir, la t�cnica ha sido aplicada seg�n lo recomendado por la IPO. En segundo lugar, las t�cnicas que han sido adaptadas para poder ser aplicadas en los desarrollos �giles. Hemos identificado en la literatura que las t�cnicas de usabilidad m�s adoptadas en las actividades de Evaluaci\'on de los PDSA son: Evaluaci�n Heur�stica \cite{Anwar2014}\cite{Faulring2012}\cite{Losada2012}\cite{Wale-Kolade2014}\cite{Kane2003}\cite{Sohaib2011}, Pensar en Voz Alta \cite{Kane2003}\cite{Losada2012}\cite{Losada2013b}\cite{Sohaib2011}, Test de Usabilidad \cite{Ambler2006}\cite{Wale-Kolade2014}\cite{Sy2007}\cite{Wolkerstorfer2008}\cite{Anwar2014}\cite{Ferreira2007}\cite{Hudson2003}\cite{Larusdottir2010}\cite{Oevad2015}\cite{Hussain2009b}\cite{Beyer2004}\cite{Dayton2009}\cite{Hussain2009b}\cite{Hussain2009c}\cite{Hussain2009d}\cite{Illmensee2009}\cite{Kushniruk2015}\cite{McGinn2013}\cite{Nielsen2012}\cite{Wolkerstorfer2008} y Focus Groups \cite{Anwar2014}\cite{Losada2012}\cite{Losada2013b}\cite{Jia2012}\cite{Oevad2015}\cite{Cajander2013}. 

Encontramos t�cnicas que se han adoptado en los desarrollos �giles de las dos maneras, es decir, pura y con adaptaciones. El tipo de adopci�n (pura o adaptada) depende de las caracter��sticas y recursos particulares de cada proyecto. La t�cnica de Test de Usabilidad se ha adoptado con adaptaciones en la mayor��a de los casos, seg�n los estudios primarios. Muchos practicantes de equipos �giles indican que es muy dif�cil realizar tests de usabilidad de la forma tradicional debido a los cronogramas ajustados de los \textit{sprints} \cite{DaSilva2015}. Por lo tanto, hemos identificado adaptaciones que proponen realizar tests de usabilidad sobre maquetas de papel o prototipos de baja fidelidad \cite{Hussain2009d}\cite{Beyer2004}, las cuales son menos formales y permiten tener retroalimentaci�n de los usuarios durante el desarrollo de un \textit{sprint} o iteraci�n. Nos llama la atenci�n la t�cnica Test de Usabilidad porque en la gran mayor�a ha sido adaptada (70\%). Hemos identificado que los autores adoptan tests de usabilidad m�s informales para cumplir con las condiciones de los PDSA. El 30\% restante, la t�cnica Test de Usabilidad se ha adoptado pura. Son tres las estrategias que identificamos permiten adoptar esta t�cnica con adaptaciones. En primer lugar, realizar Test de Usabilidad con los usuarios para evaluar cambios cada 2 o 3 \textit{sprints} con la ayuda de un equipo externo dedicado a temas de usabilidad \cite{Sy2007}\cite{Wale-Kolade2014}. En segundo lugar, definir Test de Usabilidad de bajo costo o micro tests. Un micro test \cite{Nielsen2012} es una prueba de usabilidad en l�nea, a menudo sin moderador. Esto implica realizar un n�mero limitado de pruebas, de 3 a 5 pruebas para un determinado conjunto de tareas que se ha desarrollado durante una iteraci�n en particular. Adem�s, el test de usabilidad de bajo costo \cite{Kushniruk2015} es un m�todo donde las pruebas se pueden llevar a cabo r�pidamente, a bajo costo y en cualquier lugar, lo que permite realizar tests en cualquier etapa del ciclo de desarrollo �gil. En tercer lugar, evaluar la usabilidad por medio de Tests sobre prototipos de baja y alta fidelidad \cite{Beyer2004}\cite{Hussain2009d}. Esta estrategia permite evaluar una funcionalidad con el usuario sin necesidad de tener software funcional y permite realizar ciclos de revisi�n m�s cortos de manera m�s informal. Las formas de los prototipos pueden incluir prototipos de papel, \textit{screenshots}, transparencias de \textit{PowerPoint} o maquetas en \textit{html}.\par

La disciplina de la IPO estipula que la Evaluaci�n Heur�stica es una inspecci�n del dise�o de la interfaz, realizada por especialistas de usabilidad, que verifica el cumplimiento de principios de usabilidad reconocidos \cite{Shneiderman1997}. Sin embargo, no es com�n que existan especialistas de usabilidad en el equipo �gil, por tanto se requiere trabajar con equipos de dise�adores para suplir estas carencias \cite{Anwar2014}. Una heur�stica es un conjunto de directrices dadas a los evaluadores para identificar muchos problemas de usabilidad. Es mejor utilizarlas temprano en la fase de dise�o porque es m�s f�cil solucionar muchos de los problemas de usabilidad que surgen. Sohaib y Kahn \cite{Sohaib2011} identificaron, por medio de entrevistas, que los equipos �giles utilizan las heur�sticas como una lista de chequeo a evaluar durante las pruebas de aceptaci�n que se deben realizar en XP.

En la t�cnica de Pensar en Voz Alta, los usuarios son alentados a "pensar en voz alta", manteniendo un mon�logo sobre lo que est�n haciendo mientras lo hacen. El prop�sito de esto es acceder a lo que los usuarios est�n pensando y cu�les son sus intenciones cuando usan el sistema \cite{Constantine1999}. Losada y colegas \cite{Losada2012}\cite{Losada2013b} proponen aplicar la t�cnica tal cual lo prescribe la IPO en su m�todo �gil llamado \textit{InterMod}. Kane \cite{Kane2003} propone una versi�n "simplificada" de pensar en voz alta con la modificaci�n que se puede utilizar para evaluar prototipos de papel en lugar de evaluar el sistema software ya terminado.

Por �ltimo, Focus Groups ha sido considerada como la t�cnica m�s utilizada por equipos �giles que aplican el proceso \textit{Scrum} seg�n reportan los trabajos de Jia y colegas \cite{Jia2012} y Ovad y Larsen \cite{Oevad2015}. Esta t�cnica tal cual lo prescribe la IPO se utiliza en los PDSA para validar las funcionalidades liberadas en cada \textit{sprint}.

Resumiendo, hemos identificado dos principales condiciones desfavorables para aplicar las t�cnicas de evaluaci�n de usabilidad en los PDSA. En primer lugar, est�s t�cnicas requieren de especialistas de usabilidad o DCU que permitan aplicar la t�cnica, es por ello que se proponen el trabajo colaborativo entre equipos �giles y equipos de DCU. En segundo lugar, las t�cnicas de evaluaci�n de usabilidad requieren de planeaci�n de tiempo y recursos para poder aplicar la t�cnica, lo que conlleva a que est�n muy limitadas a los tiempos cortos de los PDSA, en estos casos se plantean t�cnicas m�s informales que puedan ser aplicadas en ciclos m�s cortos.



\section{Conclusiones}\label{sec-conclusions}

En esta investigaci�n determinamos qu� t�cnicas de usabilidad se est�n adoptando en los proyectos de desarrollo �gil y analizamos c�mo se est�n adoptando. Adem�s, identificamos c�mo algunos de los PDSA est�n adaptando las t�cnicas para poder ser adoptadas. Las t�cnicas relacionadas con las actividades de Evaluaci�n m�s utilizadas son Test de Usabilidad, Evaluaci�n Heur�stica, Pensar en Voz Alta y Focus Groups. Por un lado, muchas t�cnicas de usabilidad relacionadas con las actividades de Evaluaci�n pueden ser adoptadas tal cual lo prescribe la IPO (por ejemplo, Focus Groups), mientras que otras requieren de adaptaciones para poder ser adoptadas en algunos proyectos �giles (por ejemplo, Evaluaci�n Heur�stica y Test de Usabilidad). Sin embargo, estas adaptaciones no son generalizadas ni prescriptivas, lo que hace necesario un mayor esfuerzo de investigaci�n al respecto y determinar que t�cnicas de usabilidad pueden ser incorporadas en las actividades de Evaluaci�n de los PDSA. Seg�n los resultados de nuestro SMS, parece que no existen propuestas formalizadas de incorporaci�n de las t�cnicas de usabilidad en los PDSA que establezcan gu�as de c�mo incorporar cada t�cnica. Nuestra investigaci�n busca realizar un aporte para intentar resolver este problema.\par

Identificamos dos adaptaciones principales que la comunidad de desarrollo �gil est� realizando a las t�cnicas de la IPO relacionadas con la actividad de Evaluaci�n, para poder ser incorporadas en los PDSA. La primera adaptaci�n consiste en realizar las actividades de las t�cnicas de evaluaci�n de la usabilidad colaborando con equipos DCU y trabajando en sprints paralelos. La segunda adaptaci�n, consiste en reducir costos y esfuerzo de planeaci�n de la t�cnica para generar planes m�s informales que permitan ser aplicadas en ciclos cortos.\par

Nuestros resultados sugieren que la comunidad de desarrollo �gil est� adoptando t�cnicas de usabilidad en sus proyectos de desarrollo, con especial inter�s en las t�cnicas de evaluaci�n de la usabilidad. Existen propuestas para aplicarlas tanto puras como con modificaciones, pero no hay un consenso de como ser� la mejor manera de adaptarlas. Esto puede deberse a la falta de una propuesta general, prescriptiva y sistem�tica que permita a los equipos de desarrollo �giles adoptar t�cnicas de usabilidad en sus proyectos de desarrollo.\par

Como trabajos futuros, consideramos ampliar las b�squedas de art�culos para cubrir otras BBDD, como, por ejemplo, SpringerLink y ScienceDirect. El objetivo es aumentar el n�mero de trabajos relevantes recuperados, para poder encontrar otras propuestas de aplicaci�n de t�cnicas de usabilidad, ya que hemos identificado que hay un gran inter�s en este tema dentro de la comunidad cient�fica. Adem�s, como trabajos futuros, vamos a adaptar t�cnicas de usabilidad relacionadas con las actividades de evaluaci�n para su incorporaci�n en un proyecto de desarrollo �gil real. A largo plazo, nuestra investigaci�n considera la creaci�n de un marco de integraci�n de t�cnicas de usabilidad que gu�e a los practicantes �giles en c�mo adaptar sistem�ticamente estas t�cnicas y poder reducir los tiempos de evaluaci�n a fin de cumplir con los tiempos reducidos de las iteraciones �giles.\par



\section*{Agradecimientos}
%This work was supported by DIUDA 22316 research project of Universidad de Atacama, Chile and partially supported by the \textquotedblleft Laboratorio Industrial en Ingenier�a del Software Emp�rica (LI2SE)\textquotedblright, and the \textquotedblleft An�lisis Experimental de la Efectividad de Test Driven Development en Funci�n de las Caracter�sticas Particulares de los Programadores a Nivel Personal\textquotedblright~research projects of Universidad de las Fuerzas Armadas ESPE, Ecuador.

%Este trabajo ha sido subvencionado por el proyecto de investigaci�n denominado: "Laboratorio Industrial en Ingenier�a del Software Emp�rica (LI2SE)" de la ESPE y adem�s cont� con la financiaci�n del "Proyecto DIUDA 22316" de la Universidad de Atacama.

%La presente investigaci�n ha sido financiada por los proyectos del Ministerio de Educaci�n, Cultura y Deporte de Espa�a FLEXOR (TIN2014-52129-R) y TIN2014-60490-P, el proyecto eMadrid-CM (S2013 / ICE-2715) y el \textquotedblleft Proyecto DIUDA 22316 \textquotedblright de la Universidad de Atacama. Adem�s, cont� con la financiaci�n del proyecto de investigaci�n \textquotedblleft Laboratorio Industrial en Ingenier�a del Software Emp�rica (LI2SE) \textquotedblright de la ESPE.

La presente investigaci�n ha sido financiada por los proyectos del Ministerio de Educaci�n, Cultura y Deporte de Espa�a FLEXOR (TIN2014-52129-R) y TIN2014-60490-P, el proyecto eMadrid-CM (S2013 / ICE-2715), el \textquotedblleft Proyecto DIUDA 22316 \textquotedblright de la Universidad de Atacama, y el proyecto de investigaci�n \textquotedblleft Laboratorio Industrial en Ingenier�a del Software Emp�rica (LI2SE) \textquotedblright de la ESPE.


%References
\bibliographystyle{splncs03}
\bibliography{Bibliografia}


\end{document}